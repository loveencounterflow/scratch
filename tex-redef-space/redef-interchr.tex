\documentclass{article}

\usepackage{multicol}
\usepackage{fontspec}
\newfontface{\mktsFontfileSunexta}{sun-exta.ttf}[Path=/home/flow/io/jizura-fonts/fonts/]
\newcommand{\cn}{\mktsFontfileSunexta}
% \setmainfont{Helvetica}
\newcommand{\mktsInterCjkChrSpace}{\hskip -0.4pt plus 3pt minus 1pt}
% \newcommand{\mktsInterCjkChrSpace}{\hskip -0.4pt plus 0pt minus 1pt}

% http://tex.stackexchange.com/questions/10248/font-selection-in-xetex-for-specific-characters
% ftp://ftp.mpi-sb.mpg.de/pub/tex/mirror/ftp.dante.de/pub/tex/info/xetexref/xetex-reference.pdf
\XeTeXinterchartokenstate=1
\chardef\CharNormal=0
\chardef\CharCjk=1
\chardef\CharCjkPunctA=2
\chardef\CharCjkPunctB=3
\chardef\CharBound=255

% \XeTeXinterchartoks\CharNormal\CharNormal={\mktsInterCjkChrSpace}
\XeTeXinterchartoks\CharNormal\CharCjk={\mktsInterCjkChrSpace}
\XeTeXinterchartoks\CharCjk\CharNormal={\mktsInterCjkChrSpace}
% \XeTeXinterchartoks\CharCjk\CharBound={---}
% \XeTeXinterchartoks\CharBound\CharCjk={---}
\XeTeXinterchartoks\CharCjkPunctA\CharCjk={\mktsInterCjkChrSpace}
\XeTeXinterchartoks\CharCjkPunctB\CharCjk={\mktsInterCjkChrSpace}
\XeTeXinterchartoks\CharCjk\CharCjkPunctA={\mktsInterCjkChrSpace}
\XeTeXinterchartoks\CharCjk\CharCjkPunctB={\mktsInterCjkChrSpace}
\XeTeXinterchartoks\CharCjk\CharCjk={\mktsInterCjkChrSpace}

% \XeTeXupwardsmode=1
% \XeTeXtracingfonts=1
% \newXeTeXintercharclass\ExclamClass
% \XeTeXcharclass 33=\ExclamClass
% \XeTeXinterchartoks\ExclamClass\ExclamClass={---}

\begin{document}
\begin{multicols}{3}

Computers are much pickier and less flexible about spelling than humans; thus, hackers need to be very precise when talking about characters, and have developed a considerable amount of verbal shorthand for them. Every character has one or more names — some formal, some concise, some silly. Common jargon names for ASCII characters are collected here. See also individual entries for bang, excl, open, ques, semi, shriek, splat, twiddle, and Yu-Shiang Whole Fish.
Brick Tea
{\cn{}x x x x x x
紧压茶是为了长途运输和长时间保存方便,将茶压缩干燥,压成方砖状或块状,为了防止途中变质,一般紧压茶都是用红茶或黑茶制作。
紧压茶是为了长途运输和长时间保存方便,将茶压缩干燥,压成方砖状或块状,为了防止途中变质,一般紧压茶都是用红茶或黑茶制作。
紧压茶是为了长途运输和长时间保存方便,将茶压缩干燥,压成方砖状或块状,为了防止途中变质,一般紧压茶都是用红茶或黑茶制作。
}
Brick Tea Computers are much pickier and less flexible about spelling than humans; thus, hackers need to be very precise when talking about characters, and have developed a considerable amount of verbal shorthand for them. Every character has one or more names — some formal, some concise, some silly. Common jargon names for ASCII characters are collected here. See also individual entries for bang, excl, open, ques, semi, shriek, splat, twiddle, and Yu-Shiang Whole Fish.



\end{multicols}
\end{document}
